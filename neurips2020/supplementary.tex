\appendix
\section{Proof that a CNN with correlations in the weights converges to a GP\label{app:netsor}}
{  % This bracket defines the scope for commands:
  \newcommand{\Gva}{\mathsf{G}}
  \newcommand{\Hva}{\mathsf{H}}
  \newcommand{\Ava}{\mathsf{A}}
  \newcommand{\MatMul}{{\texttt{MatMul}}}
  \newcommand{\LinComb}{{\texttt{LinComb }}}
  \newcommand{\Nonlin}{{\texttt{Nonlin }}}
  \newcommand{\Netsor}{{\textsc{Netsor }}}

In this section, we formally prove that a CNN with correlated weights converges
to a Gaussian process in the limit of infinite width. Using the \Netsor
programming language due to \citet{yang2019wide}, the proof can be completed in
one step: describe a CNN with correlated weights in \Netsor.

For the reader's convenience, we informally recall the \Netsor programming language
\citep{yang2019wide} and the key property of its programs (Corollary~\ref{corollary:netsor-gp}). The outline of our presentation here also closely follows
\citet{yang2019wide}. Readers familiar with \Netsor should skip to
\cref{sec:netsor-program}, where we show the program that proves Lemma~\ref{lemma:correlated-weights-nn}.


We write $[n]$ to mean the set $\{1,\dots,n\}$.

\subsection{The \Netsor programming language}
There are three types of variables: $\Gva(n)$-vars, $\Ava(n_1,
n_2)$-vars, and $\Hva(n)$-vars. Each of these have one or two
parameters, which are the widths we will take to infinity. For a
given index in $[n]$ (or $[n_1],[n_2]$), each of these variables is a \emph{scalar}. To
represent vectors that do not grow to infinity, we need to use collections of variables.

\begin{itemize}
\item[$\Gva$-vars] (Gaussian-vars) are $n$-wise \emph{approximately}
i.i.d. and Gaussian. That is, there can be
correlations between $\Gva$-vars, but only within a single index $i \in 1,\dots,n$.
$\Gva$-vars will converge in
distribution to an $n$-wise independent and identically distributed Gaussian in the limit of $n \to \infty$, if
all widths are $n$.

\item[$\Ava$-vars] represent matrices,
like those used in a dense neural network. Their entries are always i.i.d.
Gaussian with with zero mean, even for finite instantiations of the program.
There are no correlations between different $\Ava$-vars, or elements of the same $\Ava$-var.

\item[$\Hva$-vars] represent variables that become $n$-wise i.i.d. (not necessarily
Gaussian) in the
infinite limit. $\Gva$ is a subtype of $\Hva$, so all $\Gva$-vars are also $\Hva$-vars.
\end{itemize}

We indicate the type of a variable, or each variable in a
collection, using ``$\text{var} : \text{Type}$''.


\begin{definition}[Netsor program]
A \textsc{Netsor} program consists of:
\begin{itemize}
  \item[\bf Input:]
 A set of $\Gva$-vars or $\Ava$-vars.

 \item[\bf Body:]
New variables can be defined using the following rules:
\begin{enumerate}
  \item[\MatMul] $\Ava(n_1, n_2) \times \Hva(n_2) \to \Gva(n_1)$. Multiply an
    an i.i.d. Gaussian matrix times an i.i.d. vector, which becomes a Gaussian
    vector in the limit $n_2 \to \infty$.
  \item[\LinComb] Given constants $\alpha_1,\dots,\alpha_K$, and $\Gva$-vars
    $x_1,\dots,x_K : \Gva(n_1)$, their linear combination $\sum_{k=1}^K \alpha_k
    x_k$ is a $\Gva$-var.
  \item[\Nonlin] applying an elementwise nonlinear function $\phi : \Reals^k \to
    \Reals$, we map
    several $\Gva$-vars to one $\Hva$-var. 
\end{enumerate}

\item[\bf Output:]
A tuple of scalars $(v^\tp x_1/\sqrt{n_1}, \;\dots, \; v^\tp
x_K/\sqrt{n_K})$. The variable $v : \Gva(n)$ is an input $\Gva$-var not used
elsewhere in the program. Each $x_k : \Hva(n_k)$ is an $\Hva$-var.
\end{itemize}
\end{definition}

\subsection{The output of a \Netsor program converges to a Gaussian process}
\begin{definition}[Controlled function]
  A function $\phi: \Reals^k \to \Reals$ is \emph{controlled} if
  \[ \abs{\phi(\vx)} \le C\, \text{exp} \bra{\bracket{\|}{\|}{\vx}_2^\bra{2-\epsilon}}
    + c \]
  for 
  $C,c,\epsilon > 0$, where $\|\cdot\|_2$ is the L2 norm.\end{definition}

Intuitively, this means that the function $\phi$ grows
more slowly than the rate at which the tail of a Gaussian decays. Recall that the tail of a
mean zero, identity covariance Gaussian decays as 
$\mathcal{N}\bra{\vx \vbar {\boldsymbol{0}}, \eye} \propto \exp\bra{-\bracket{\|}{\|}{\vx}_2^2}$.

\begin{assumption}
  All nonlinear functions $\phi(\cdot)$ in the \Netsor program are controlled.
\end{assumption}

\begin{assumption}[Distribution of $\Ava$-var inputs]
  Each element $W_{\prevchan,\chan} \in A^i(n, n)$ in each input $\Ava$-var is
  sampled from the zero-mean, i.i.d. Gaussian $\NormDist{0,
    \sigma_\text{w}^2/n}.$
  \label{ass:avar-inputs}
\end{assumption}
\begin{assumption}[Distribution of $\Gva$-var inputs]
  For each channel $\chan \in [n]$ the vector $[x_c : x\text{ is input
  }\Gva\text{-var}]$ is drawn from a Gaussian with mean $\mu^\text{in}$ and
  covariance $\Sigma^\text{in}$. The covariance $\Sigma^\text{in}$ may be
  singular. The $\Gva$-var that corresponds to the output is sampled
  independently from all other $\Gva$-vars.
  \label{ass:gvar-inputs}
\end{assumption}

\citet{yang2019wide} goes on to prove the \Netsor master theorem, from which
the corollary of interest follows.

\begin{corollary}[Corollary~5.5, abridged, \cite{yang2019wide}]
  Fix a \Netsor program with controlled nonlinearities, and draw its inputs
  according to assumptions \ref{ass:avar-inputs} and \ref{ass:gvar-inputs}. For simplicity, fix
  the widths of all the variables to $n$. The program outputs are $(v^\tp x_1/\sqrt{n}, \;\dots, \; v^\tp
  x_K/\sqrt{n})$, where
  each $x_k$ is an $\Hva$-var, and
    $v$ is a $\Gva$-var independent from all others with variance $\sigma^2_\text{v}$.
  Then, as $n \to \infty$, the output tuple
  converges in distribution to a Gaussian $\NormDist{{\boldsymbol{0}}, \vK}$.
  \label{corollary:netsor-gp}
\end{corollary}

The covariance $\vK$ can be calculated recursively using the rules given by
\citet{yang2019wide}. Informally, they consist of recursively calculating the covariances
of the
program's $\Gva$-vars, assuming at every step that the $\Gva$-vars are $n$-wise
i.i.d. and Gaussian. This is the approach we employ in \cref{sec:correlated-weights}.

% \subsection{Description in \Netsor of a CNN with correlations in the weights\label{sec:netsor-program}}
% The canonical way to represent
% convolutional filters in \Netsor \citep[\Netsor program 4]{yang2019wide} is to
% use one $\Ava$-var for every spatial location
% of the weights. That is, if our convolutional patches have size $\patchw{\ell} \times \patchh{\ell}$,
% we define the input $\layerW{\ell}_\patchidx: \Ava(\layerC{\ell+1}, \layerC{\ell})$ for
% all $\patchidx \in [\patchsize{\ell}]$. But $\Ava$-vars have to be independent
% of each other, so how can we add correlation in the weights? We apply the
% correlation separately, using a \LinComb{} operation. For this, we will use the
% following well-known lemma.
% \begin{lemma}
%   Fix $\vSigma$, a positive semi-definite (PSD) covariance matrix. Let $u_1,\dots,u_K \sim
%   \NormDist{0, 1}$ be a collection of i.i.d. Gaussian random variables. Let
%   $\vL$ be a lower-triangular matrix such that $\vL\vL^\tp = \vSigma$, which
%   always exists if $\vSigma$ is PSD. Let $w_i \eqdef \sum_{k=1}^iL_{ik}u_k$ for
%   $i \in [K]$, equivalently
%   $\vw = \vL\vu$. Then $\vw \sim \Normal{{\boldsymbol{0}}, \vSigma}$.
%   \label{lemma:L}
% \end{lemma}
% \begin{proof}
%   $\{w_i\}$ are Gaussian because they are a linear transformation of $\{u_k\}$.
%   Calculate their moments to see they are distributed as stated.
% \end{proof}



% {  % program commands scope
%   \newcommand{\Input}[1]{\State{\textbf{Input} #1}}
%   \newcommand{\Output}[1]{\State{\textbf{Output} #1}}
%   \newcommand{\CommentC}[1]{\State{\textit{// #1}}}
%   \newcommand{\layerBs}[1]{B^{(#1)}}
% \begin{algorithm}
%   \caption{\Netsor description of the CNN in \cref{fig:fancy-cnn} with
%     correlated weights}
% \begin{algorithmic}
%   \CommentC{Program for $M$ training + test points, $\vX_1,\dots,\vX_M$.}
%   \Input{$\layerAs{1}_{\patch}\bra{\vX_m} : \Gva(\layerC{1})$ for $\patch \in
%     \sqb{\layersize{1}}$ and $m \in [M]$. First layer activations for all inputs}
%   \Input{$\layerW{\ell}_{\patchidx} : \Ava\bra{\layerC{\ell}, \layerC{\ell+1}}$
%     for $\patchidx \in [patchsize]$ and $\ell \in \{1, 2, 3\}$}

%   \CommentC{Construct the second layer's independent activations}
%   \For{$m \in [M]$, $\mu \in \sqb{\layersize{1}}$, $\patchidx \in \sqb{\patchsize{1}}$}
%     \State{\MatMul: $\layerBs{2}_\patchidx(\vX_m) \eqdef
%       \layerW{1}_{\patchidx}\layerAs{1}_\mu(\vX_m) : \Gva(\layerC{2})$}
%   \EndFor

%   \CommentC{Correlate the activations according to $\priorWcov{2}$}
%   \For{$m \in [M]$, $\mu \in \sqb{\layersize{1}}$, $\patchidx \in \sqb{\patchsize{1}}$}
%   \State{\MatMul: $\layerBs{2}_\patchidx(\vX_m) \eqdef
%     \layerW{1}_{\patchidx}\layerAs{1}_\mu(\vX_m) : \Gva(\layerC{2})$}
%   \EndFor
% \end{algorithmic}
% \end{algorithm}
% } % End program scope


\begin{lemma}[CNN with correlations in the weights converges to a GP]
\label{lemma:correlated-weights-nn}
\end{lemma}

}% End command scope

\section{Experimental details \label{app:experimental}}

%%% Local Variables:
%%% mode: latex
%%% TeX-master: "main"
%%% End:
